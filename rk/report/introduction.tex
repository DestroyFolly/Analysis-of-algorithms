\centerchapter{Введение}
\hspace{\parindent}
Сегодня компьютерам требуется выполнять все более трудоемкие вычисления. Одним из возможных решений, позволяющих увеличить производительность компьютера, является паралллельное программирование. Параллельные вычисления на одном устройстве реализуются при помощи многопоточности --- возможности процессора одновременно выполнять несколько потоков, поддерживаемых ОС.

Целью работы является исследование параллельного программирования на примере алгоритма нахождения определителя матрицы методом миноров. 
Для решения поставленной цели, необходимо выполнить следующие задачи:
\begin{enumerate}
	\item Описать метод многопоточной обработки данных;
	\item Разработать алгоритм нахождения определителя матрицы методом миноров;
	\item Разработать параллельный алгоритм нахождения определителя матрицы методом миноров;
	\item Сравнить разработанные алгоритмы.
\end{enumerate}