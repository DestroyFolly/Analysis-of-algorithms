\averchapter{Аналитическая часть}
\hspace{\parindent}В этом разделе будет описан принцип многопоточной обработки данных и алгоритм многопоточной обработки данных.

\section{Многопоточная обработка данных}
Многопоточность - это способность процессора выполнять одновременно несколько потоков, используя ресурсы одного процессора. Поток представляет собой последовательность инструкций, которые могут выполняться параллельно с другими потоками в рамках одного процесса.
Процесс - это программа, которая находится в состоянии выполнения. При запуске программы или приложения создается процесс. Процесс может состоять из одного или нескольких потоков, где каждый поток выполняет задачи, необходимые для работы приложения. Процесс завершается, когда все его потоки завершают свою работу.

В многопоточных программах необходимо учитывать, что если потоки запускаются последовательно и передают управление друг другу, то не получится полностью использовать потенциал многопоточности и получить выигрыш от параллельной обработки задач. Чтобы достичь максимальной эффективности, потоки должны выполняться параллельно, особенно для независимых по данным задач.



\section{Алгоритм нахождения опредилителя матрицы методом миноров}
Матрица --- это набор чисел, записанный в виде прямоугольной таблицы.
Строки и столбцы матрицы можно рассматривать как векторы. Матрица с одинаковым количеством строк и одинаковым количеством столбцов называется квадратной. Для обозначения элемента матрицы \textit{A}, стоящего в \textit{i-той} строке и в \textit{j-ом} столбце используется запись \textit{A}[i][j].

Минором \textit{Mij} элемента \textit{aij} матрицы \textit{A} \textit{n}\textit{-го} порядка называется определитель $(n-1)$\textit{-го} порядка, полученного из исходного определителя вычеркиванием \textit{i}\textit{-ой} строки и \textit{j}\textit{-го} столбца \cite{minor}.
Алгебраическим дополнением \textit{Aij} элемента \textit{aij} матрицы \textit{A} \textit{n}\textit{-го} порядка называется число, равное произведению минора \textit{Mij} на $(-1)^{i+j}$: 
\begin{equation}
	Aij=(-1)^{i+j}\cdot M_{ij}.
\end{equation}

Таким образом определитель $n$\textit{-го} порядка вычисляется с помощью метода понижения порядка --- по формуле \textit{detA}=$\sum\limits_{j=1}^n$\textit{aij}\textit{Aij} (\textit{i} фиксировано) --- разложение по $i$\textit{-ой} строке.

\section{Вывод}
В этом разделе был описан принцип многопоточной обработки данных и алгоритм многопоточной обработки данных.


