\chapter*{Заключение}
\addcontentsline{toc}{chapter}{Заключение}

В результате исследования было получено, что сложность всех алгоритмов равняется O($N^{3}$), а по используемой памяти алгоритмы практически не отличаются.
Также было установленно, что при размере матриц, большем 30, небходимо использовать оптимизированный алгоритм умножения матриц по Винограду, так как данный алгоритм работает быстрее стандартного алгоритма в 1.3 раза засчет замены операции равно и плюс на операцию плюс-равно. При этом стандартный алгоритм медленнее алгоритма Винограда в 1.2 раза из-за того, что в алгоритме Винограда часть вычислений происходит заранее. 
Кроме того алгоритм Винограда предпочтительно использовать для умножения матриц четных размеров, так как указанный алгоритм работает в 1.2 раза быстрее, чем на матрицах с нечетным размером. Это связано с проведением дополнительных вычислений для крайних строк и столбцов.

Цель, поставленная перед началом работы, была достигнута. В ходе лабораторной работы были решены следующие задачи:

\begin{itemize}
	\item были изучены классический алгоритм, алгоритм Винограда и его оптимизированная версия умножения матриц;
	\item были разработаны изученные алгоритмы;
	\item был проведен сравнительный анализ реализованных алгоритмов;
	\item был подготовлен отчет о выполненной лабораторной работе.
\end{itemize}
