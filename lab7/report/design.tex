\chapter{Конструкторская часть}

В данном разделе будут описаны используемые структуры, представлен листинг псевдокода и описан метод оценивания количества сравнений.


\section{Описание используемых структур}
При реализации алгоритмов будут использованы следующие структуры данных:
\begin{itemize}
	\item node --- содержит значение и указатель на левого и правого потомка;
	\item tree --- содержит корневой узел.
\end{itemize}


\section{Разработка алгоритмов}

Псевдокод для поиска в сбалансированном и несбалансированном бинарном дереве поиска представлен в листинге \ref{lst:pseudo}.

Входные данные: root - корень дерева, key - искомое значение.
                
Выходные данные: node - узел, содержащий искомое значение.	
\begin{lstlisting}[label=lst:pseudo,caption=Псевдокод для поиска в бинарном дереве поиска]			
	node = root
	while(True):
	if node is None or node.key == key then
	return node
	end if
	
	if node.key > key then
	node = node.left
	else
	node = node.right
	end if
	end while
\end{lstlisting}

\section{Оценка количества сравнений}
Для дальнейших замеров количества сравнений необходимо определить, что является лучшим и худшим случаем для разработанного алгоритма.

Лучшим случаем является нахождение числа в корне дерева.
В таком случае потребуется 1 сравнение для нахождения искомого числа.

Худшим случаем в данной реализации будет являться отсутствие узла в дереве, так как в таком случае возможно будет выполнить $h + 1$ сравнений, где $h$ --- высота дерева, в то время как при нахождении искомого элемента на максимальной высоте дерева количество сравнений равно $h$.

\section*{Вывод}

В данном разделе были описаны используемые структуры, представлен листинг псевдокода и описан метод оценивания количества сравнений.