\chapter{Технологическая часть}

В данном разделе будут указаны средства реализации, будут представлены листинги кода, а также функциональные тесты.

\section{Средства реализации}

Реализация данной лабораторной работы выполнялась при помощи языка программирования Python \cite{python}. Выбор ЯП обусловлен простотой синтаксиса, большим числом библиотек и эффективностью визуализации данных.

Замеры времени проводились при помощи функции process\_time из библиотеки time \cite{python-time}.

\section{Сведения о модулях программы}

Программа состоит из следующих модулей:

\begin{itemize}
	\item main.py - главный файл программы, предоставляющий пользователю меню для выполнения основных функций;
	\item matrix.py - файл, содержащий функции работы с матрицами;
	\item salesman.py - файл, содержащий метды решения задачи коммивояжера;
	\item time\_test.py - файл, содержащий функции замеров времени работы указанных алгоритмов;
	\item graph\_result.py - файл, содержащий функции визуализации временных характеристик описанных алгоритмов.
\end{itemize}

\section{Реализация алгоритмов}

В листинге \ref{lst:search} представлена реализация алгоритма полного перебора.

\lstinputlisting[label=lst:search,caption=Реализация алгоритма полного перебора, firstline=1,lastline=19]{img/brute_force.py}

В листинге \ref{lst:search} представлена реализация муравьиного алгоритма с элитными муравьями.

\lstinputlisting[label=lst:search,caption=Реализация муравьиного алгоритма с элитными муравьями, firstline=1,lastline=74]{img/ants.py}

В листинге \ref{lst:search} представлена реализация алгоритма обновления феромонов с учетом элитных муравьев.

\lstinputlisting[label=lst:search,caption=Реализация алгоритма обновления феромонов с учетом элитных муравьев, firstline=1,lastline=11]{img/up_pher.py}

В листинге \ref{lst:search} показана реализация алгоритма алгоритма выбора следующего города.

\lstinputlisting[label=lst:search,caption=Реализация алгоритма алгоритма выбора следующего города, firstline=1,lastline=12]{img/cnl.py}


\section{Функциональные тесты}

В таблице~\ref{tbl:func-tests} приведены результаты функционального тестирования реализованных алгоритмов.
Все тесты пройдены успешно.

\begin{table}[H]
	\caption{Функциональные тесты}
	\label{tbl:func-tests}
	\centering
	\begin{tabular}{|c|c|c|}
		\hline
		Матрица смежности & Полный перебор & Муравьиный алгоритм \\ \hline
		$ \begin{pmatrix}
			0 &  4 &  2 &  1 & 7 \\
			4 &  0 &  3 &  7 & 2 \\
			2 &  3 &  0 & 10 & 3 \\
			1 &  7 & 10 &  0 & 9 \\
			7 &  2 &  3 &  9 & 0
		\end{pmatrix}$ &
		15, [0, 2, 4, 1, 3] &
		15, [0, 2, 4, 1, 3] \\ \hline
		$ \begin{pmatrix}
			0 & 1 & 2 \\
			1 & 0 & 1 \\
			2 & 1 & 0	
		\end{pmatrix}$ &
		4, [0, 1, 2] &
		4, [0, 1, 2] \\ \hline
		$ \begin{pmatrix}
			0 & 15 & 19 & 20 \\
			15 &  0 & 12 & 13 \\
			19 & 12 &  0 & 17 \\
			20 & 13 & 17 &  0
		\end{pmatrix}$ &
		64, [0, 1, 2, 3] &
		64, [0, 1, 2, 3] \\ \hline
	\end{tabular}
\end{table}

\section{Вывод}

Были реализованы функции алгоритмов решения задачи коммивояжера. Было проведено функциональное тестирование указанных функций.