\chapter{Технологическая часть}

В данном разделе будут указаны средства реализации, будут представлены листинги кода, а также функциональные тесты.

\section{Средства реализации}

Реализация данной лабораторной работы выполнялась при помощи языка программирования Python \cite{python}. 
В текущей лабораторной работе было необходимо определить процессорное время, затрачиваемое на выполнение программы, а также построить графики. 
Все необходимые инструменты для этого имеются в выбранном языке программирования.

Замеры времени проводились при помощи функции process\_time из библиотеки time \cite{python-time}.

\section{Сведения о модулях программы}

Программа состоит из следующих модулей:

\begin{itemize}
	\item main.py - главный файл программы, предоставляющий пользователю меню для выполнения основных функций;
	\item sort.py - файл, содержащий функции сортировок массива;
	\item arr.py - файл, содержащий функции создания массива различного типа и работы с массивом;
	\item time\_test.py - файл, содержащий функции замеров времени работы сортировок;
	\item graph\_result.py - файл, содержащий функции визуализации временных характеристик алгоритмов сортировок.
\end{itemize}

\section{Реализация алгоритмов}

Реализации алгоритмов сортировок выбором, бусинами и блочной сортировки представлены на листингах \ref{lst:selection_sort}, \ref{lst:bead_sort}, \ref{lst:bucket_sort}.

\begin{center}
\captionsetup{justification=raggedright,singlelinecheck=off}
\begin{lstlisting}[label=lst:selection_sort,caption=Алгоритм сортировки выбором]
def selection_sort(arr, size):
    for i in range(size - 1):
        min_element = arr[i]
        min_index = i

        for j in range(i + 1, size):
            if arr[j] < min_element:
                min_element = arr[j]
                min_index = j
        
        arr[i], arr[min_index] = arr[min_index], arr[i]

	return arr
\end{lstlisting} 
\end{center}

\begin{center}
\captionsetup{justification=raggedright,singlelinecheck=off}
\begin{lstlisting}[label=lst:bead_sort,caption=Алгоритм сортировки бусинами]
def bead_sort(input_list, size):
	return_list = []
	transposed_list = [0] * max(input_list)
	for num in input_list:
		transposed_list[:num] = [n + 1 for n in transposed_list[:num]]
	for i in range(size):
		return_list.append(sum(n > i for n in transposed_list))
	return return_list

\end{lstlisting}
\end{center}

\clearpage

\begin{center}
\captionsetup{justification=raggedright,singlelinecheck=off}
\begin{lstlisting}[label=lst:bucket_sort,caption=Алгоритм блочной сортировки]
def bucket_sort(array, length):
	largest = max(array)
	size = largest / length

	buckets = [[] for i in range(length)]

	for i in range(length):
		index = int(array[i] / size)
		if index != length:
			buckets[index].append(array[i])
		else:
			buckets[length - 1].append(array[i])

	for i in range(len(array)):
		buckets[i] = sorted(buckets[i])

	result = []
	for i in range(length):
		result = result + buckets[i]

	return result
\end{lstlisting}
\end{center}

\section{Функциональные тесты}

В таблице \ref{tbl:func-tests} приведены функциональные тесты для функций, реализующих алгоритмы сортировок. Все тесты пройдены успешно.

\begin{table}[h]
	\begin{center}
		\caption{\label{tbl:func-tests} Функциональные тесты}
		\begin{tabular}{|c|c|c|}
			\hline
			Входной массив & Ожидаемый результат & Результат \\ 
			\hline
			$[1, 2, 3, 4, 5, 6]$ & $[1, 2, 3, 4, 5, 6]$  & $[1, 2, 3, 4, 5, 6]$\\
			$[5, 4, 3, 2, 1]$  & $[1, 2, 3, 4, 5]$ & $[1, 2, 3, 4, 5]$\\
			$[4, 5, 3, 7, 8]$  & $[3, 4, 5, 7, 8]$  & $[3, 4, 5, 7, 8]$\\
			$[17]$  & $[17]$  & $[17]$\\
			$[]$  & $[]$  & $[]$\\
			\hline
		\end{tabular}
	\end{center}
\end{table}

\section*{Вывод}

Были реализованы функции алгоритмов сортировки выбором, бусинами и блочной. Было проведено функциональное тестирование указанных функций.