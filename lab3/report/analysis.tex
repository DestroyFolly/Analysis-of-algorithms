\chapter{Аналитическая часть}

В данном разделе будут описаны алгоритмы сортировки выбором, бусинами и блочной сортировки.

\section{Сортировка выбором}

Сортировка выбором состоит из следующих шагов:

\begin{itemize}
	\item выбирается элемент неотсортированной части последовательности с наименьшим значением;
	\item выбранный элемент меняется местами с элементом, стоящим на первой позиции в неотсортированной части. Обмен не нужен, если это и есть минимальный элемент;
	\item повтор шагов 1 и 2 до тех пор, пока не останется только наибольший элемент.
\end{itemize}

\section{Блочная сортировка}

Алгоритм сортировки, который работает путем распределения элементов массива по нескольким сегментам. Затем каждый сегмент сортируется индивидуально, либо с использованием другого алгоритма сортировки, либо путем рекурсивного применения блочной сортировки.

Алгоритм работает следующим образом.
\begin{itemize}
	\item Пусть $l$ – минимальный, а $r$ – максимальный элемент массива. Разобьем элементы на блоки, в первом будут элементы от $l$ до $l + k$, во втором – от $l + k$ до $l + 2k$ и т.д., где $k = (r$ – $l) / \textup{количество блоков}$.
	\item Все получившиеся блоки сортируются другим алгоритмом сортировки.
	\item Выполняется слияние блоков в единый массив.
\end{itemize}

\section{Сортировка бусинами}

Алгоритм сортировки  бусинами, работает путем создания матрицы и последующего распределения элементов исходного массива. Каждое число записывается количество раз равное самому числу. Затем матрица трансплонируется и подсчитывается количество вхождений. Таким образом мы получаем отсортированный числа исходного массива \cite{beads}.

Метод ограничен, прежде всего применим к натуральным числам, т.е. можно сортировать только положительные числа. 
Можно сортировать и целые, но это запутаннее - отрицательные числа придется обрабатывать отдельно отрицательные от положительных.


\section*{Вывод}

Были рассмотрены следующие алгоритмы сортировки: выбором, бусинами и блочная. 
Для указанных алгоритмов необходимо получить теоретическую оценку и доказать её экспериментально.