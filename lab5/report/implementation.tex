\chapter{Технологическая часть}
\hspace{\parindent}В данном разделе будут рассмотрены средства реализации, а также представлены листинги реализаций алгоритма линейной обработки данных и алгоритма конвейерной обработки данных.

\section{Средства реализации}
\hspace{\parindent}В данной работе для реализации был выбран язык программирования C++. Его возможностей достаточно для измерения процессорного времени и реализации алгоритмов.


\hspace{\parindent}Время работы было замерено с помощью функции
system\_clock::now()\cite{proc-time} из библиотеки std::chrono.


\section{Реализация алгоритмов}
\hspace{\parindent}В листингах \ref{lst:linear}-\ref{lst:conveyor} представлены реализации алгоритмов линейной и конвейерной обработки даных.
\clearpage

\begin{center}
	\captionsetup{skip=0pt,justification=raggedright,singlelinecheck=off}
    \begin{lstlisting}[label=lst:linear,language=Python, caption=Алгоритм линейной обработки данных]
void linear_execution(int count, size_t size, bool is_print)
{
	time_now = 0;
	for (int i = 0; i < count; i++)
	{
		matrix_t matrix = matrix_generate(size);
		matrix.det = matrix_determinant(matrix, 0, matrix.size, 1);
		matrix_dump(matrix);
	}
}
    \end{lstlisting}
\end{center}


\begin{center}
	\captionsetup{skip=0pt,justification=raggedright,singlelinecheck=off}
    \begin{lstlisting}[label=lst:conveyor,language=Python,caption=Алгоритм конвейерной обработки данных]
void parallel_execution(int count, size_t size)
{
	std::queue<int> q1;
	std::queue<matrix_t> q2;
	std::queue<matrix_t> q3;
	queues_t queues = {.q1 = q1, .q2 = q2, .q3 = q3};
	for (int i = 0; i < count; i++)
		q1.push(size);
	std::thread threads[3];
	threads[0] = std::thread(stage1_parallel, std::ref(q1), std::ref(q2), std::ref(q3));
	threads[1] = std::thread(stage2_parallel, std::ref(q1), std::ref(q2), std::ref(q3));
	threads[2] = std::thread(stage3_parallel, std::ref(q1), std::ref(q2), std::ref(q3));
	for (int i = 0; i < 3; i++)
		threads[i].join();
}
\end{lstlisting}
\end{center}
\clearpage

\begin{center}
	\captionsetup{skip=0pt,justification=raggedright,singlelinecheck=off}
	\begin{lstlisting}[label=lst:gen,language=Python,caption=Алгоритм генерации квадратной матрицы]
matrix_t matrix_generate(size_t size)
{
	std::vector<std::vector<int>> tmp_data;
	tmp_data.resize(size);
	for (size_t i = 0; i < size; i++)
	tmp_data[i].resize(size);
	matrix_t matrix;
	matrix.size = size;
	matrix.data = tmp_data;
	matrix.det = 0;
	for (size_t i = 0; i < matrix.size; i++)
	for (size_t j = 0; j < matrix.size; j++)
	matrix.data[i][j] = std::experimental::randint(1, 10);
	return matrix;
}
	\end{lstlisting}
\end{center}
\begin{center}
	\captionsetup{skip=0pt,justification=raggedright,singlelinecheck=off}
	\begin{lstlisting}[label=lst:det,language=Python,caption=Алгоритм вычисления детерминанта матрицы]
int matrix_determinant(matrix_t &matrix, int start, int end, int newDegree)
{
	int det = 0;
	int degree = newDegree;
	int size = matrix.size;
	if(size == 1)
	return matrix.data[0][0];
	else if(size == 2)
	return matrix.data[0][0]*matrix.data[1][1] - matrix.data[0][1]*matrix.data[1][0];
	else
	{
		for(int j = start; j < end; j++)
		{
			matrix_t copy = matrix_copy(matrix);
			matrix_WithoutRowAnColumn(copy, 0, j);
			det += degree * matrix.data[0][j] * matrix_determinant(copy, 0, copy.size, 1);;
			degree = -degree;
		}
	}
	return det;
}
	\end{lstlisting}
\end{center}
\clearpage
\begin{center}
\captionsetup{skip=0pt,justification=raggedright,singlelinecheck=off}
\begin{lstlisting}[label=lst:dump,language=Python,caption=Алгоритм записи матрицы и детерминанта в файл]
void matrix_dump(matrix_t matrix)
{
	ofstream logf(LOG, ios::app);
	if (logf.is_open())
	{
		for (size_t i = 0; i < matrix.size; i++)
		{
			for (size_t j = 0; j < matrix.size; j++)
			logf << matrix.data[i][j] << " ";
			logf << "\n";
		}
		logf << "\n\nDet: " << matrix.det << "\n----------\n";
		logf.close();
	}
}
\end{lstlisting}
\end{center}

\section{Функциональные тесты}
\hspace{\parindent}В таблице  \ref{tbl:testing} представлены функциональные тесты для программы.
\begin{table}[h]
	\begin{center}
		\captionsetup{skip=0pt,justification=raggedright,singlelinecheck=off}
		\caption{Функциональные тесты}
		\label{tbl:testing}
		\begin{tabular}{|c|c|c|c|}
			\hline
			Матрица & Линейная & Конвейерная & Эталон\\
			\hline
			 $\begin{pmatrix}
				0 & 1 & 2\\
				1 & 0 & 3\\
				2 & 4 & 0
			\end{pmatrix}$ &
			14 &
			14 &
			14 \\
			\hline
			$\begin{pmatrix}
				4 & 7 & 2 & 1 & 4\\
				4 & 2 & 3 & 5 & 2\\
				4 & 3 & 7 & 9 & 1\\
				3 & 7 & 9 & 2 & 9\\
				7 & 9 & 3 & 1 & 7
			\end{pmatrix}$ &
			-867 & -867 & -867 \\
			\hline
			$\begin{pmatrix}
				1 & 3 & 2 & 2\\
				7 & 4 & 2 & 5\\
				4 & 6 & 1 & 4\\
				2 & 6 & 9 & 7
			\end{pmatrix}$ &
			71 & 71 & 71 \\
			\hline
		\end{tabular}	
	\end{center}
\end{table}