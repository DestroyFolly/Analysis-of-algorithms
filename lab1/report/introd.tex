\chapter*{Введение}
\addcontentsline{toc}{chapter}{Введение}


Важная часть программирования состоит в операциях работы со строками, которые часто используются для различных задач, таких как создание обычных статей или записей в базу данных. В связи с этим возникает несколько важных задач, для которых требуются алгоритмы сравнения строк. В данной работе будет рассмотрено обсуждение этих алгоритмов.. 
Подобные алгоритмы используются при:
\begin{itemize}
	\item исправлении ошибок в тексте, предлагая заменить введенное слово с ошибкой на наиболее подходящее;
    \item поиске слова в тексте по подстроке;
    \item сравнении целых текстовых файлов. \newline
\end{itemize}



\textbf{Целью данной работы} является изучение, реализация и исследование алгоритмов нахождения расстояний Левенштейна и Дамерау--Левенштейна. 
Для достижения поставленной цели необходимо выполнить следующие задачи:
\begin{itemize}
	\item изучить и реализовать алгоритмы нахождения расстояния Левенштейна и Дамерау-Левенштейнаа;
	\item изучить и реализовать матричую реализацию, а также реализацию с использованием кеша в виде матрицы для алгоритма Левенштейна;
    \item провести сравнительный анализ алгоритмов Левенштейна и Дамерау-Левенштейна, а также сравнение рекурсивной и матричной реализаций, матричной реализации и реализаций с кешом алгоритма Левенштейнаа;
    \item подготовить отчет о лабораторной работе.
\end{itemize}
