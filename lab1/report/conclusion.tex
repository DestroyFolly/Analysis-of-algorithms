\chapter*{Заключение}
\addcontentsline{toc}{chapter}{Заключение}

По результатам исследования было установлено, что при увеличении длины строк для алгоритмов Левенштейна и Дамерау-Левенштейна время выполнения растет в геометрической прогрессии. Рекомендуется отдавать предпочтение алгоритму Левенштейна, так как он в два раза быстрее, при длине строк более шести символов. Однако, наилучшие показатели по времени достигаются при использовании матричной реализации алгоритма Левенштейна и рекурсивной реализации с кешем, которые обеспечивают 12-кратное превосходство по времени работы уже при длине строки в 4 символа, за счет сохранения промежуточных вычислений. В то же время следует учитывать, что матричные реализации требуют значительного объема памяти при большой длине строк.


Цель, которая была поставлена в начале лабораторной работы была достигнута, а также в ходе выполнения лабораторной работы были решены следующие задачи:

\begin{itemize}
	\item были изучены и реализованы алгоритмы нахождения расстояния Левенштейна и Дамерау-Левенштейна;
	\item были также изучены матричная реализация, а также реализация с использованием кеша в виде матрицы для алгоритма Левенштейна;
    \item проведен сравнительный анализ алгоритмов Левенштейна и Дамерау-Левенштейна, а также сравнение рекурсивной и матричной реализаций, матричной реализации и реализаций с кешом алгоритма Левенштейна;
	\item подготовлен отчет о лабораторной работе.
\end{itemize}