\chapter*{Введение}
\addcontentsline{toc}{chapter}{Введение}

По мере развития вычислительных систем программисты столкнулись с
необходимостью производить параллельную обработку данных для улучшения отзывчивости системы, ускорения производимых вычислений и рационального использования вычислительных мощностей. Благодаря развитию
процессоров стало возможным использовать один процессор для выполнения нескольких параллельных операций, что дало начало термину <<Многопоточность>>.

Целью данной лабораторной работы является изучение принципов и получение навыков организации параллельного выполнения операций.

