\chapter{Аналитическая часть}

В данном разделе будут описаны алгоритмы кластеризации: кластеризация методом k-средних и параллельный алгоритм кластеризации методом k-средних.

\section{Кластеризация алгоритмом k-средних}

Первые применения алгоритма k-средних были описаны в работе Джеймса МакКуина в 1967 году. 
При заранее известном числе кластеров k алгоритм k-средних начинает с некоторого начального разбиения документов и уточняет его,
оптимизируя целевую функцию – среднеквадратичную ошибку кластеризации \cite{book}.


Действие алгоритма начинается с выбора k начальных центров кластеров.
Обычно исходные центры кластеров выбираются случайным образом. 
Затем каждый документ присваивается тому кластеру, чей центр является наиболее близким
документу, и выполняется повторное вычисление центра каждого кластера как центроида, или среднего своих членов. 
Такое перемещение документов и повторное вычисление центроидов кластеров продолжается до тех пор, пока не будет достигнуто условие остановки. 
Условием остановки может служить следующее: 
\begin{itemize}
	\item достигнуто пороговое число итераций;
	\item центроиды кластеров больше не изменяются;
	\item достигнуто пороговое значение ошибки кластеризации.
\end{itemize}
На практике используют комбинацию критериев остановки, чтобы одновременно ограничить время работы
алгоритма и получить приемлемое качество.



\section{ Параллельный алгоритм кластеризации методом k-средних}

Чтобы уменьшить врмея выполнения алгоритма, следует распараллелить ту часть алгоритма, которая содержит высчитывает расстояние от элемента массива до центра кластеризации.
Вычисление результата не зависит от результата подсчета для других элементов.
Поэтому можно распараллелить часть кода, где проходят эти действия.
Каждый поток будет выполнять вычисления расстояния для элемента массива.


\section{Вывод}

Были рассмотрены алгоритмы кластеризации методом k-средних и возможность его оптимизации с помощью распараллеливания потоков.
Была расмотрена технология параллельного программирования и организация взаимодействия параллельных потоков.