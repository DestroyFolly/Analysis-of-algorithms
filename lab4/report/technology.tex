\chapter{Технологическая часть}

В данном разделе будут указаны средства реализации, будут представлены реализации алгоритмов, а также функциональные тесты.

\section{Средства реализации}

Реализация данной лабораторной работы выполнялась при помощи языка программирования Python.  В текущем лабораторном задании необходимо определить процессорное время, затрачиваемое на выполнение программы, а также ввести многопоточное программирование. Все необходимые инструменты для этого имеются в выбранном
языке программирования\cite{python}.

Замеры времени проводились при помощи функции process\_time из библиотеки time \cite{python-time}.
Многопоточное программирование было реализованно с помощью модуля ThreadPoolExecutor.

\section{Сведения о модулях программы}

Программа состоит из следующих модулей:

\begin{itemize}
	\item main.py - главный файл программы, предоставляющий пользователю меню для выполнения основных функций;
	\item func.py - файл, содержащий функции кластеризации и сравнения времени.
\end{itemize}

\section{Реализация алгоритмов}

Алгоритм кластеризации методом k-средних и его параллельная реализация приведены в листингах \ref{lst:standard}, \ref{lst:optimized}.
\clearpage

\begin{center}
\captionsetup{justification=raggedright,singlelinecheck=off}
\begin{lstlisting}[label=lst:standard,caption=Алгоритм кластеризации методом k-средних]
def clusterization(array, k):
	n = len(array)
	dim = len(array[0])
	cluster = [[0 for i in range(dim)] for q in range(k)]
	cluster_content = [[] for i in range(k)]
	for i in range(dim):
		for q in range(k):
			cluster[q][i] = randint(0, MAX_CLUSTER_VAL)
	cluster_content = data_distribution(array, cluster, n, k, dim)
	privious_cluster = copy.deepcopy(cluster)

	while 1:
		cluster = cluster_update(cluster, cluster_content, dim)
		cluster_content = data_distribution(array, cluster,n,k,dim)
		if cluster == privious_cluster:
			break
		privious_cluster = copy.deepcopy(cluster)
	visualisation_3d(cluster_content)
\end{lstlisting} 
\end{center}
\clearpage

\begin{center}
\captionsetup{justification=raggedright,singlelinecheck=off}
\begin{lstlisting}[label=lst:optimized,caption=Параллельная реализация алгоритма кластеризации]
def asinc_clusterization(array, cluster, n, k, dim):
    for i in range(n):
	min_distance = float('inf')
	situable_cluster = -1
	for j in range(k):
		distance = 0
		for q in range(0,dim, 3):

		with ThreadPoolExecutor() as executor:
			res = executor.map(summing(array[i][q], cluster[j][q] ))
			for num in res:
				distance += num
		distance = distance ** (1 / 2)
		if distance < min_distance:
			min_distance = distance
			situable_cluster = j

	cluster_content[situable_cluster].append(array[i])
\end{lstlisting}
\end{center}

\section{Функциональные тесты}

В таблице \ref{tbl:func_test} приведены функциональные тесты для функций, реализующих алгоритмы кластеризации методом k-средних. Все тесты пройдены успешно. \newpage

\begin{table}[h]
	\begin{center}
		\begin{threeparttable}
			\captionsetup{justification=raggedright,singlelinecheck=off}
			\caption{\label{tbl:func_test} Функциональные тесты}
			\begin{tabular}{|c@{\hspace{7mm}}|c@{\hspace{7mm}}|c@{\hspace{7mm}}|c@{\hspace{7mm}}|c@{\hspace{7mm}}|c@{\hspace{7mm}}|}
				\hline
				 Массив входных чисел & Количествово кластеров & Распределение \\ 
				\hline
				$\begin{pmatrix}
					1 & 2 & 3 & 4 & 5
				\end{pmatrix}$ &
				$\begin{pmatrix} 
					2 \\ 
				\end{pmatrix}$ &
				$\begin{pmatrix}
					1 & 2 \\
					3 & 4 & 5
				\end{pmatrix}$ \\ \hline
				$\begin{pmatrix}
					1 & 2 & 3 & 4 & 5
				\end{pmatrix}$ &
				$\begin{pmatrix} 
					5 \\ 
				\end{pmatrix}$ &
				$\begin{pmatrix}
					1 \\
					2 \\
					3 \\
					4 \\
					5 
				\end{pmatrix}$ \\ \hline
				$\begin{pmatrix}
					1 & 2 & 3 & 4 & 5
				\end{pmatrix}$ &
				$\begin{pmatrix} 
					1 \\ 
				\end{pmatrix}$ &
				$\begin{pmatrix}
					1 & 2 & 3 & 4 & 5
				\end{pmatrix}$ \\ \hline				
			\end{tabular}
		\end{threeparttable}
	\end{center}
	
\end{table}

\section{Вывод}

Были реализованы функции алгоритмов кластеризции. Было проведено функциональное тестирование указанных функций.